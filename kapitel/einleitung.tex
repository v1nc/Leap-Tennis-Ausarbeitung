%========================================================================================
% TU Dortmund, Informatik Lehrstuhl VII
%========================================================================================

\chapter{Einleitung}
\label{Einleitung}
1 Einleitung
Das vorliegende Arbeit beschäftigt sich mit dem Abschlussprojekt des Visual Computing Fachprojekts zur Realisierung eines virtuellen Tennis Spiels, bei dem 
die Steuerung durch Leap Motion umgesetzt wird.
In dem Projekt wurden einige Projekte, welche zuvor im Fachprojekt vorgestellt wurden, kombiniert. Hierbei wurde vor allem Leap Motion für die Realisierung des Schlägers genutzt und des weiteren
Grundkenntnisse in OpenGL um die benötigten Objekte zu visualisieren.




\section{Motivation}
\label{Motivation_und_Hintergrund}
%
1.1 Motivation 
Für die letzte Phase des Fachprojekts wurde die Vorgabe gestellt, ein eigenständiges Projekt in einer kleinen Gruppe zu entwickeln, wobei wir dabei
das bereits Gelernte kreativ einsetzen sollten, damit die erlernten Kompentenzen aufgefrischt und gefestigt werden.
Die Idee für die wir uns nach kurzer Beratungszeit entschieden haben sollte ein Tennis Spiel sein, wobei die Hand als Schläger fungiert und die Erkennung der Hand
durch Leap Motion gewährleistet wird. 
Das Projekt besteht aus hinreichend vielen Problemen, welche wir gleichmäßig aufteilten. Die Teilprobleme haben unterteilt in die physikalischen Einflüsse, wobei
wir großen Wert auf einen realistischen Spielfluss gelegt haben, das Rendering, um die gebrauchten Objekte zu erstellen und die Steuerung, welches vorallem die korrekte Erfassung der Hand
beinhaltete. 
Die Regeln des Spieles sind sehr schnell zu erlernen und sprechen jede mögliche Zielgruppe an, da die Regeln leicht verständlich und die Steuerung sehr intuitiv sind. %deswegen haben wir großen Wert auf ein an




\newpage
\section{Problemstellung}
\label{Aufbau_der_Arbeit}
%

Um das Tennis Spiel zu realisieren haben wir uns intensiv mit folgenden drei Teilproblemem beschäfigt. Da sie teilweise nicht aufeinander aufbauten, hat es sich angeboten parallel an denen zu arbeiten.\\
Für die Steuerung des Spiels musste der Leap Motion Kontroller korrekt angebunden werden, sodass zusätzliche Daten bezüglich der Hand gewonnen werden konnte, welche beispielweise für die Ausrichtung des Schlägers wichtig war.
Durch Anbindung des Leap Motion Kontrollers haben wir das Ziel verfolgt die Intuitivität des Spielers anzusprechen, da er zum Spielen lediglich seine Hand braucht.\\
Die Objekte, welche visualisiert werden sollten, mussten gerendert werden. Dies beinhaltete unter anderem eine Box in der das Spiel abläuft, einen Ball und einen Schläger. Außerdem wurden
einige Extras eingebunden, um dem ganzen Projekt mehr Farbe zu geben.\\
Der dritte Themenbereich umfasst die ganze Physik und die Berechnungen welche diese mit sich führt. Dabei haben wir Wert darauf gelegt Faktoren die in der Realität vorhanden sind, wie 
zum Beispiel die Schwerkraft oder ähnliches in unser Projekt zu implementieren, um dem Spieler eben dieses Gefühl von Intuitivität und Realität wiederzugeben. 
Die Kollisionserkennung und Berechnung fällt unter diesen Teilbereich, wobei wieder versucht wurde die genannten Faktoren zu realisieren.


