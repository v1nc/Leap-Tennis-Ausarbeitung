%========================================================================================
% TU Dortmund, Informatik Lehrstuhl VII
%========================================================================================

\chapter{Ergebnisse und Diskussion}
\label{Kapitel 5}
%
Um die Umsetzung des Fachprojekts zu analysieren und ausreichend zu diskutieren, wurden Testpersonen befragt, nachdem sie eine Zeit lang das Spiel getestet haben. Folgende Fragen wurden ihnen danach gestellt:
\begin{enumerate}
	\item \glqq Welche Probleme haben Sie bei der Steuerung bemerkt ?\grqq
	\item \glqq Wie haben Sie die Steuerung im Vergleich zum normalen Tischtennis empfunden ?\grqq
	\item \glqq Als wie realitätsnah haben Sie die Bewegung des Balls empfunden ?\grqq
	
\end{enumerate}

\section{Graphische Ausgabe}
\label{Kapitel_5_-_Unterkapitel_1}
Die graphische Ausgabe wurde von allen Testpersonen für gut empfunden. Probleme, die bei der Steuerung auftraten,  wurden durch die Kamerabewegung beim Bewegen der Hand und die nahe Ansicht auf die Szene verstärkt. Eine Testperson antwortete auf Frage eins: \glqq Da die Kamera sich mit der Hand bewegt, ist es manchmal schwer den Ball zu treffen.\grqq. Die Bewegung der Kamera wäre nicht notwendig, wenn die Szene aus größerer Entfernung betrachtet werden würde.
\section{Steuerung}
\label{Kapitel_5_-_Unterkapitel_2}
Beim Bedienen der Anwendung traten  die meisten Probleme auf. Eine Person antwortete auf Frage zwei: \glqq Das tatsächliche Gefühl, Tischtennis zu spielen, ist nicht vorhanden. Der Ball wird mit der flachen Hand geschlagen, nicht mit einem Objekt, was in der Hand gehalten wird.\grqq \ Zur Bestimmung der Position des Schlägers sollte nicht der Handmittelpunkt, sondern der Mittelpunkt eines gedachten Schlägers in der Hand des Spielers verwendet werden. Dadurch könnte der Spieler das Gefühl erhalten, er schlage den Ball mit einem Schläger.
\section{Physik}
\label{Kapitel_5_-_Unterkapitel_3}
Einige Testpersonen empfanden, dass die Geschwindigkeit des Balls nach dem Schlagen zu hoch war und zu schnell abnahm. Um den Flug des Balls noch realistischer zu gestalten, könnte die berechnete Gravitation erhöht werden, außerdem sollte die Geschwindigkeit des Balls nach dem er getroffen wurde geringer sein.